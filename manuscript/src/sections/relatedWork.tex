%! Author = vanchondo
%! Date = 05/08/2022

\section{Related Works}

Con el auge de los algoritmos de Inteligencia Artificial y los algoritmos de
visión por computadora, ya existen otros estudios similares al presentado,
tales como:
\begin{itemize}
      \item Deep Facial Expression Recognition: A Survey\cite[]{li2020deep}.
      \item The first facial expression recognition and analysis
            challenge\cite[]{valstar2011first}.
      \item Facial expression recognition from video sequences: temporal and static
            modeling.\cite[]{cohen2003facial}.
      \item Facial expression recognition based on Local Binary Patterns: A comprehensive
            study\cite[]{shan2009facial}.
      \item Facial Expression Recognition by De-Expression Residue
            Learning\cite[]{yang2018facial}.
      \item Facial expression recognition using facial movement
            features\cite[]{zhang2011facial}.
      \item Facial expression recognition with identity and emotion joint
            learning\cite[]{li2018facial}.
      \item Annotation: Development of facial expression recognition from childhood to
            adolescence: Behavioural and neurological
            perspectives\cite[]{herba2004annotation}.
      \item Does facial expression recognition provide a toehold for the development of
            emotion understanding?\cite[]{strand2016does}.
      \item Emotion-modulated attention improves expression recognition: A deep learning
            model\cite[]{barros2017emotion}.
      \item Deep-emotion: Facial expression recognition using attentional convolutional
            network\cite[]{minaee2021deep}.
      \item Laplacian Nonlinear Logistic Stepwise and Gravitational Deep Neural
            Classification for Facial Expression
            Recognition\cite[]{kumari4096801laplacian}.
      \item A facial expression recognizer using modified ResNet-152\cite[]{xu2022facial}.
      \item Evoker: Narrative-based Facial Expression Game for Emotional Development of
            Adolescents\cite[]{hong2022evoker}.

\end{itemize}
Estos estudios cuentan con bases similares al aquí presentado, la gran diferencia es que estos
no generan resultados para un lapso de tiempo, ni tampoco un resultado final sobre el impacto
de lo que está viviendo o viendo la persona.