%! Author = vanchondo
%! Date = 05/09/2022

\section{Results}

En esta primera etapa del software, se realizaron pruebas con personas de
diferentes edades, las categorías que vamos a manejar son:
\begin{itemize}
	\item Infantes: Entre 6 y 11 años.
	\item Adolescentes: Entre 12 y 17 años.
	\item Jóvenes: Entre 18 y 26 años.
	\item Adultos: Entre 27 y 59 años.
\end{itemize}
La cantidad de personas que participaron de las diferentes categorías son:
\begin{table}[h!]
	\centering
	\caption{Número de personas que participaron por categoría. {\label{tab:stats_category_num}}}
	\resizebox{4cm}{!}{
    \begin{tabular}{ |c|c| }
        \hline
        \toprule
        Categoría    & Cantidad \\
        \midrule
        Infantes     & 55       \\
        Adolescentes & 53       \\
        Jóvenes      & 34       \\
        Adultos      & 27       \\
        \bottomrule
        \hline
    \end{tabular}
}
\end{table}

Después de haber realizado pruebas con diferentes personas de las diferentes
categorías en condiciones apropiadas (Iluminación adecuada, poco movimiento, y
fotografías de alta calidad) se obtuvieron los siguientes resultados:

\begin{table}[h!]
	\centering
	\caption{Número de personas que participaron por categoría. {\label{tab:stats_facial_recognition_results}}}
	\resizebox{8cm}{!}{
  \begin{tabular}{ |c|c|c|c|c|c| }
    \hline
    \toprule
    Categoría       & Cantidad & Aciertos & Errores & \% de Aciertos & \% de Errores \\
    \midrule
    Infantes        & 55       & 37       & 18      & 67.27\%        & 32.72\%       \\
    Adolescentes    & 53       & 40       & 13      & 75,47\%        & 24.52\%       \\
    Jóvenes         & 34       & 26       & 8       & 76.47\%        & 23.52\%       \\
    Adultos         & 27       & 22       & 5       & 81.48\%        & 18.51\%       \\
                    &          &          &         &                &               \\
                    &          &          &         &                &               \\
    Total Promedio: & 42.25    & 31.25    & 11      & 73.96\%        & 26.03\%       \\
    \bottomrule
    \hline
  \end{tabular}
}
\end{table}

Podemos observar que se obtuvo un promedio global de 73.96\% de aciertos y de
26.03\% de errores. Estos resultados son aceptables para esta primera etapa del
sistema. \\En posteriores versiones del sistema, después de seguir entrenando
la red con más personas de distintas edades y de distintas regiones se espera
lograr un porcentaje de aciertos cercano al 90\%. \\Para lograr esto, se está
planificando realizar visitas a escuelas de diferentes niveles que nos permitan
recopilar la información sin perturbar las actividades diarias de los alumnos.
Así mismo también se está colaborando con empresas del sector privado para
realizar más captura de información.