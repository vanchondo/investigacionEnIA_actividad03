%! Author = vanchondo
%! Date = 10/06/2020

\section{Methodology}

La metodología por utilizar será CRISP-DM.\@
\begin{enumerate}
      \item Comprensión del negocio (Business Understanding). \\Los objetivos de este
            proyecto son:
            \begin{itemize}
                  \item Poder reconocer e interpretar las expresiones faciales de un grupo de personas
                        durante la presentación de un discurso, comercial o producto nuevo.
                  \item Una vez procesada la información, poder entregar al cliente un
            \end{itemize}

      \item Comprensión de datos (Data Understanding) Para poder comprender e interpretar
            la información recolectada, se diseñará un clasificador el cual será entrenado
            utilizando un data set existente llamado AffectNet\cite[]{mollahosseini2017affectnet}, el cual
            cuenta con las siguientes clasificaciones:
            \begin{itemize}
                  \item Neutral
                  \item Happy
                  \item Sad
                  \item Surprise
                  \item Fear
                  \item Disgust
                  \item Anger
                  \item Contempt
                  \item None
                  \item Uncertain
                  \item Non-Face
            \end{itemize}

      \item Preparación de datos (Data Preparation). \\Para preparar los datos se realizan
            los siguientes pasos:
            \begin{itemize}
                  \item La información será capturada con una o mas cámaras las cuales tomaran fotos al
                        grupo muestra de personas durante la presentación.
                  \item Se procesan esas fotos para separar las caras detectadas en una foto en
                        imágenes diferentes.
                  \item Se aplican técnicas de limpieza sobre la imagen para eliminar la existencia de
                        ruido.
            \end{itemize}

      \item Modelado (Modeling) \\Las imágenes se categorizan por lapsos de tiempo, por
            ejemplo, cada 5 segundos, este tiempo es personalizable, para así poder obtener
            los cambios de los rostros faciales del publico de una manera promediada y
            saber mejor el impacto que se esta obteniendo debido a la presentación.

      \item Evaluación (Evaluation) \\Para la evaluación, se van a ejecutar múltiples
            experimentos con grupos de personas diversos, por ejemplo, un grupo de
            estudiantes de primaria, secundaria y prepa en donde se les presente el tráiler
            de una nueva película. Al finalizar, se tiene que hacer una encuesta para
            obtener información básica sobre que fue lo que sintieron durante la
            presentación. De esta forma podremos verificar si los resultados obtenidos por
            el sistema son similares al obtenido de la encuesta. Algo a tener en mente es
            que para las personas es mas fácil quedarse con el sentimiento de lo ultimo que
            vieron, entonces si el inicio de la presentación les pareció muy interesante y
            el final muy aburrido, podríamos llegar a obtener resultados diferentes, dicho
            esto, se recomienda que para las evaluaciones se utilicen presentaciones de
            periodos cortos y que no abusen de las emociones del publico.

      \item Implementación (Deployment) \\Para la implementación se planea hacer uso de
            servicios en la nube para realizar todo el procesamiento de la información. Y
            el uso de un equipo moderado para la captura de las imágenes tales como cámaras
            y una computadora para poder subir esta información a nuestro servicio para
            iniciar el procesamiento.
\end{enumerate}